\chapter{Conclusion and Open Questions}
\label{chap:conclusion}

\section{Summary of Findings}

In this thesis, we introduced \emph{Sharp-It}, a multi-view diffusion model designed to enhance the quality and detail of 3D objects generated by latent diffusion models. Sharp-It successfully addresses the trade-off between geometric controllability provided by direct 3D generative models and the visual fidelity achievable by multi-view-based reconstruction methods.

Our experiments demonstrated that Sharp-It consistently produces higher-quality and geometrically coherent outputs compared to several baseline enhancement methods, including optimization-based and image-conditioned diffusion approaches. By starting from coarse yet geometrically plausible initial shapes provided by Shap-E, our method mitigates common artifacts such as inconsistent textures, unrealistic geometry, and the "Janus" effect prevalent in purely image-based generative methods. Additionally, we validated Sharp-It's capabilities through diverse practical applications, such as semantic appearance editing, controlled shape generation, and high-quality object refinement from textual descriptions.

\section{Practical Implications}

Sharp-It’s framework has significant practical implications across multiple industries reliant on digital 3D asset creation, notably in gaming, animation, augmented reality (AR), and virtual reality (VR). The ability to rapidly produce refined, consistent, and high-quality 3D assets from simple text prompts or coarse 3D models streamlines content creation workflows, reduces manual refinement effort, and accelerates production pipelines.

Furthermore, Sharp-It’s combination of multi-view refinement and text-conditioned diffusion offers intuitive tools for content creators, allowing greater artistic control and simplified workflows in creating and editing complex 3D models. By bridging the quality gap inherent in existing generative methods, our approach presents a robust solution suitable for both rapid prototyping and the production of final-quality digital assets.

\section{Limitations and Challenges}

Despite its strengths, Sharp-It currently faces several limitations and practical challenges. First, our pipeline relies heavily on the quality of sparse-view reconstruction methods used in post-processing. Consequently, the final visual quality and geometric accuracy can be constrained by the limitations inherent to the chosen reconstruction backend.

Second, although trained on diverse lighting conditions, Sharp-It lacks direct control over the illumination settings during the synthesis process, limiting user control over nuanced lighting scenarios critical in professional content production. Additionally, computational resource requirements remain relatively high, potentially limiting accessibility in resource-constrained scenarios.

\section{Open Questions and Future Directions}

Several promising avenues for future research emerge from this work:

\paragraph{Improved Sparse-view Reconstruction.}
Exploring advanced reconstruction methods capable of robustly converting multi-view images into high-quality 3D meshes would greatly enhance Sharp-It’s practical utility. Methods explicitly designed to handle Sharp-It’s refined outputs, maintaining fidelity across varied geometries and textures, would provide substantial benefit.

\paragraph{Dynamic and Deformable Objects.}
Our work has primarily addressed static 3D objects. An important open question is how multi-view diffusion approaches can extend effectively to dynamic, animated, or deformable shapes, preserving geometric coherence and temporal consistency. Future work could explore temporal conditioning and cross-frame consistency mechanisms within diffusion models to tackle these challenges.

\paragraph{Integration with Real-world Data.}
Sharp-It currently focuses on enhancing synthetic or generated data. Extending this framework to handle real-world captured 3D data, often characterized by noise, occlusions, and partial observations, presents an interesting direction. This would significantly broaden the applicability of Sharp-It, enabling refined 3D assets directly from real-world scans.

\paragraph{Fine-grained Editability and Control.}
Providing users with more intuitive and precise control over the diffusion refinement process is another critical direction. Future research might focus on methods that enable localized geometric edits, selective texture refinement, and detailed stylistic control via richer textual or visual prompts.

\paragraph{Computational Efficiency.}
Lastly, further improvements in computational efficiency, such as adopting more efficient attention mechanisms or alternative neural architectures optimized specifically for multi-view consistency, could make Sharp-It accessible to a broader range of users, including those with limited computational resources.

We believe addressing these open questions will further extend the impact of Sharp-It, enabling richer and more efficient workflows for creating visually appealing, consistent, and controllable 3D content.