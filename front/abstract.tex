% This file contains the abstract part of your thesis - in English and
% in Hebrew (within \abstractEnglish and \abstractHebrew respectively).
%
% Notes:
% - This file uses the UTF-8 character set encoding for the Hebrew
%   text not to get garbled. Keep it that way.
% - Assuming your thesis is mainly in English, Graduate School 
%   regulations mandate the following lengths for the abstracts:
%
%      Language    Min. Length   Max. Length
%     ---------------------------------------
%      English       200 words     500 words
%      Hebrew        500 words   2,000 words
%
%   so that the Hebrew abstract typically has some content from
%   the English introduction and an overview of the results, not
%   present in the English; it is not just a translation.

\abstractEnglish{

% opening sentence
Advancements in text-to-image diffusion models have led to significant progress in fast 3D content creation. 
One common approach is to generate a set of multi-view images of an object, and then reconstruct it into a 3D model.
% the problem
However, this approach bypasses the use of a native 3D representation of the object and is hence prone to geometric artifacts and limited in controllability and manipulation capabilities.
An alternative approach involves native 3D generative models that directly produce 3D representations. These models, however, are typically limited in their resolution, resulting in lower quality 3D objects.
% the focus of this work
In this work, we bridge the quality gap between methods that directly generate 3D representations and ones that reconstruct 3D objects from multi-view images.
% how are we doing it
We introduce a multi-view to multi-view diffusion model called \ourname, which takes a 3D consistent set of multi-view images rendered from a low-quality object and enriches its geometric details and texture.
The diffusion model operates on the multi-view set in parallel, in the sense that it shares features across the generated views.
A high-quality 3D model can then be reconstructed from the enriched multi-view set.
% evaluation
By leveraging the advantages of both 2D and 3D approaches, our method offers an efficient and controllable method for high-quality 3D content creation.
We demonstrate that \ourname{} enables various 3D applications, such as fast synthesis, editing, and controlled generation, while attaining high-quality assets. 


} % end of English abstract


\abstractHebrew{

% Note that certain commands don't work that well in Hebrew "mode".
% If this happens to you, try wrapping the command within a
% \textenglish{ } - that may (or may not) help.
ההתקדמות האחרונה במודלים של פיזור טקסט לתמונה האיצה משמעותית את ההתקדמות בתחום של יצירת תוכן תלת-ממדית מהירה, במיוחד על ידי הקלה על יצירת נכסים ויזואליים באיכות גבוהה באמצעים אוטומטיים. באופן מסורתי, יצירת אובייקטים תלת מימדיים מפורטים ומציאותיים דרשה מאמץ ידני משמעותי מאמנים מיומנים, וכתוצאה מכך צינור ייצור איטי ויקר. השילוב של מודלים של דיפוזיה שינה את הדינמיקה הזו, ואיפשר שיטות מהירות, מדרגיות ונגישות יותר להפקת תוכן תלת מימד.

מתודולוגיה נפוצה אחת שעלתה מההתקדמות הללו כוללת יצירת קבוצה של תמונות מרובות תצוגות המייצגות את האובייקט מזוויות שונות, ולאחר מכן שחזור תמונות אלה למודל תלת מימדי מגובש. גישת שחזור מבוססת תמונה זו ממנפת התקדמות רבת עוצמה בראייה ממוחשבת ובפוטוגרמטריה, שיכולות לפרש נקודות מבט מרובות כדי לבנות ייצוג תלת מימדי שלם. תהליך זה הפך לפופולרי יותר ויותר בשל הפשטות, האפקטיביות והתאימות שלו ליכולות המפותחות של מודלים מחוללים מבוססי תמונה.
עם זאת, שיטה רווחת זו נושאת מגבלות מהותיות משמעותיות. הבעיה העיקרית היא שהוא אינו משתמש ישירות בייצוג תלת מימדי מקורי של האובייקט. במקום זאת, היא מסתמכת אך ורק על תמונות דו מימדיות כצעדי ביניים, מה שמגביל מטבעו את יכולת השיטה לייצג פרטים גיאומטריים מורכבים במדויק. כתוצאה מכך, לעתים קרובות מתעוררים חפצים גיאומטריים, עיוותים וחוסר עקביות. בעיות אלו מחייבות לעתים קרובות תיקונים ידניים נוספים או חידוד חישוב נוסף, אשר מפחיתים את היעילות והמעשיות שבהתחלה הופכים את הגישה הזו למושכת.

יתרה מכך, ההסתמכות על ייצוגי תמונת ביניים מפחיתה את יכולת השליטה והמניפולציה הכוללת של האובייקטים התלת-ממדיים שנוצרו. מכיוון שהגיאומטריה הבסיסית משוחזרת בעקיפין באמצעות הסקה חזותית מנקודות מבט מרובות, המודלים שנוצרו בדרך כלל חסרים שליטה פרמטרית ישירה או יכולת עריכה אינטואיטיבית ברמה הגאומטרית. לפיכך, מעצבים, אנימטורים או מפתחי משחקים שרוצים לשנות במדויק את הצורה או המרקם של אובייקטים אלה נתקלים לעיתים בקשיים, שכן שינויים קטנים בתמונות עלולים לגרום לווריאציות בלתי צפויות או לא מכוונות בגיאומטריית התלת-ממד הסופית.
כדי להתמודד עם החסרונות הללו, צמחה גישה חלופית בצורה של מודלים מחוללים תלת-ממדיים, המייצרים ישירות ייצוגים תלת מימדיים של אובייקטים מקידוד סמוי או הנחיות טקסט. מודלים תלת-ממדיים מקוריים אלו מקודדים מטבעם יחסים מרחביים וגאומטריים, ומבטיחים באופן תיאורטי מבנה גיאומטרי עקבי ונטול חפצים. עם זאת, למרות היתרונות התיאורטיים שלהם, מודלים כאלה נאבקו בדרך כלל עם מגבלות משמעותיות ברזולוציה הניתנת להשגה. המורכבות של מידול ישיר וסינתזה של גיאומטריה תלת מימדית ברזולוציה גבוהה דורשת משאבי חישוב נרחבים, מה שמציב אתגרים לשמירה על יעילות ואיכות בו זמנית. כתוצאה מכך, מודלים מחוללים תלת-ממדיים אלה מניבים לעתים קרובות פלטים באיכות נמוכה יותר, המאופיינים בפרטים גיאומטריים מטושטשים או מפושטים מדי ובנאמנות טקסטורה מוגבלת.

בהתחשב בנקודות החוזק והחולשה המנוגדות של שתי הגישות - שחזור מבוסס-תמונה מרובה תצוגה המציע איכות חזותית גבוהה יותר אך יכולת שליטה ונאמנות גיאומטרית מוגבלת, וגישות יצירת תלת-ממד מקוריות המספקות עקביות גיאומטרית משופרת אך רזולוציה נמוכה יותר - נותר פער איכות משמעותי בפתרונות העדכניים הנוכחיים. גישור על פער זה דורש מתודולוגיה חדשנית הרותמת את החוזקות של שתי הפרדיגמות כדי לייצר מודלים תלת מימדיים איכותיים, ניתנים לעריכה ועקביים גיאומטרית בצורה יעילה מבחינה חישובית.
בעבודה זו, אנו מתייחסים במפורש לאתגר זה על ידי הצעת גישה חדשנית המשלבת ביעילות את התכונות הטובות ביותר של שתי השיטות. באופן ספציפי, אנו מציגים מודל דיפוזיה מרובה-תצוגות ל-רב-צפיות בשם \ourname{}. הגישה שלנו מתחילה בסט של תמונות מרובות תצוגה המעובדות מיצג תלת מימד ראשוני ואיכותי יותר. תמונות אלו מהוות בסיס ויזואלי עקבי, הלוכד מבנה גיאומטרי ראשוני של האובייקט.

\ourname{} אז פועל ישירות על ערכת תמונות מרובה-תצוגה זו, תוך שימוש בטכניקות יצירתיות מבוססות דיפוזיה כדי להעשיר את הפרטים הגיאומטריים הראשוניים ולשפר באופן משמעותי את איכות המרקם. שלא כמו שיטות שחזור תמונה מסורתיות, מודל הדיפוזיה שלנו מעבד מספר תצוגות בו זמנית ובמקביל, תוך שיתוף תכונות נלמדות על פני נקודות מבט שונות. שיתוף תכונה מקביל ועקבי זה מאפשר לדגם לאכוף ביעילות עקביות גיאומטרית על פני כל מערך ריבוי התצוגה תוך שיפור בו-זמנית של פרטי טקסטורה עדינים ומאפייני פני שטח.
החידוש הקריטי במתודולוגיה שלנו טמון באופן שבו מודל הדיפוזיה משפר בו זמנית את נקודות המבט המרובות, תוך שמירה על קוהרנטיות מרחבית ושלמות גיאומטרית. על ידי התחשבות במפורש במתאמי הראיון ובאילוצי העקביות, המודל מבטיח שהתמונות המועשרות משקפות במדויק מבנה אובייקט תלת מימדי ריאליסטי ואיכותי. לאחר מכן, ניתן לשלב תמונות מרובות תצוגה משופרות אלו בצורה חלקה בצינורות שחזור קיימים, וכתוצאה מכך מודל תלת מימד סופי משופר משמעותית עם פחות חפצים, דיוק גיאומטרי גדול יותר ואיכות חזותית משופרת משמעותית.

ההערכה המקיפה שלנו מדגימה את היעילות והמעשיות של השיטה המוצעת שלנו. מינוף היתרונות המשלימים של שיטות יצירת תלת מימד מבוססות תמונה ושיטות מקוריות, \ourname{} מציע מסגרת עוצמתית, יעילה וניתנת לשליטה ליצירת תוכן תלת מימד באיכות גבוהה. בדקנו בהרחבה את הגישה שלנו על פני תרחישים שונים של יישומים, והמחישנו את הפוטנציאל שלה לסנתז במהירות אובייקטים חדשים, לערוך בקלות גיאומטריה ומרקמים קיימים, ולהקל על משימות יצירתיות מבוקרות, כגון העברת סגנון או שינוי סמנטי.

באמצעות הניסויים שלנו, אנו מראים כי \ourname{} עולה בהרבה על גישות שחזור מבוססות תמונה מסורתיות במונחים של עקביות גיאומטרית, הפחתת חפצים ויכולת שליטה. יתרה מכך, בהשוואה לשיטות יצירת תלת-ממד ישירות, המסגרת שלנו משיגה גיאומטריה ברזולוציה גבוהה משמעותית ופרטי מרקם עשירים יותר מבלי לגרור עלויות חישוב עצומות. כתוצאה מכך, המתודולוגיה שלנו מגשרת ביעילות על פער האיכות שזוהה קודם לכן, וממצבת את עצמה ככלי רב-תכליתי ומעשי למגוון רחב של יישומים בתחומים כמו בידור, מציאות מדומה, משחקים, מסחר אלקטרוני וחינוך, שבהם ייצור מהיר ואיכותי של נכסי תלת מימד ומניפולציה מועילים מאוד.

לסיכום, על ידי הצגת גישת הדיפוזיה החדשנית שלנו מרובה-צפיות-ל-רב-צפיות, \ourname{}, אנו מספקים פתרון חזק לאחד האתגרים הדחופים ביותר ביצירת תוכן תלת-ממדית מודרנית. השיטה שלנו לא רק משלבת את העושר החזותי של גישות מבוססות תמונה עם היתרונות הגיאומטריים של מודלים מחוללים תלת מימדיים, אלא גם משיגה איכות כללית מעולה, יעילות ויכולת שליטה. בסופו של דבר, זה מאפשר צינורות ייצור נגישים, גמישים וניתנים להרחבה יותר, ומקדם באופן משמעותי את המצב המתקדם ביצירת תוכן אוטומטי בתלת מימד.z
} % end of Hebrew abstract
